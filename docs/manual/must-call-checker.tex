\htmlhr
\chapterAndLabel{Must Call Checker}{must-call-checker}

The Must Call Checker conservatively over-approximates
the set of methods that an object should call before it is de-allocated.
The checker does not enforce any rules other than subtyping;
in particular, it does not enforce that the methods are called before
objects are de-allocated.
The Must Call Checker is intended to be run as a subchecker of another checker.
The primary client
of the Must Call Checker is the Resource Leak Checker (Section~\ref{resource-leak-checker}),
which enforces that every method in a must-call obligation for an expression is called
before that expression is de-allocated.

For example, consider a \<java.io.OutputStream>.  This \<OutputStream>
might have an obligation to call the \<close()> method, to release an
underlying file resource. The type of this \<OutputStream> is
\<@MustCall(\{"close"\})>.  Or, the \<OutputStream> might not have such an
obligation, if the underlying resource is a byte array. The type of this
\<OutputStream> is \<@MustCall(\{\})>.
For an arbitrary \<OutputStream>, the Must Call
Checker over-approximates the methods that it should call by assigning it
the type \<@MustCall(\{"close"\}) OutputStream>, which can be read as ``an
OutputStream that might need to call \<close()> (but no other methods)
before it is de-allocated''.


\sectionAndLabel{Must Call annotations}{must-call-annotations}

The Must Call Checker supports these type qualifiers:

\begin{description}

\item[\refqualclasswithparams{checker/mustcall/qual}{MustCall}{String[] value}]
  gives a superset of the methods that
  must be called before the expression's value is de-allocated.
  The annotation is not refined after a method is called; an expression of
  type \<@MustCall("foo")> might or might not have already had \<foo()>
  called on it.
  The default type qualifier for an unannotated type is \<@MustCall(\{\})>.

\item[\refqualclass{checker/mustcall/qual}{MustCallUnknown}]
  represents a value with an unknown or infinite set of must-call obligations.
  It is used internally by the type system but should never be written by a
  programmer.

\item[\refqualclass{checker/mustcall/qual}{PolyMustCall}]
  is the polymorphic annotation for the Must Call type system.
  For a description of polymorphic qualifiers, see
  Section~\ref{method-qualifier-polymorphism}.

\item[\refqualclasswithparams{checker/mustcall/qual}{InheritableMustCall}{String[] value}]
  is an alias of \<@MustCall> that can only be written on a class declaration.
  It applies both to the class declaration on which it is written and also to all subclasses.
  This frees the user of the need to write the \<@MustCall> annotation on every subclass.
  See Section~\ref{must-call-on-class} for details.

\end{description}

\begin{figure}
\includeimage{mustcall}{5cm}
\caption{Part of the Must Call Checker's type
qualifier hierarchy.  The full hierarchy contains one \<@MustCall> annotation
for every combination of methods.
Qualifiers in gray are used internally by the type system but should
never be written by a programmer.}
\label{fig-must-call-hierarchy}
\end{figure}

Here are some facts about the type qualifier hierarchy, which is shown in
Figure~\ref{fig-must-call-hierarchy}.
Any expression of type \<@MustCall(\{\}) Object> also has type
\<@MustCall(\{"foo"\}) Object>.
The type \<@MustCall(\{"foo"\}) Object>
contains objects that need to call \<foo> and objects that need to call
nothing, but the type does not contain an
object that needs to call \<bar> (or both \<foo> and \<bar>).
\<@MustCall(\{"foo", "bar"\}) Object> represents all objects that need to
call \<foo>, \<bar>, both, or neither; it cannot not represent an object that needs
to call some other method.


\sectionAndLabel{Writing \<@MustCall>/\<@InheritableMustCall> on a class}{must-call-on-class}

As explained in Section~\ref{upper-bound-for-use}, a type
annotation on a type declaration means that every use of the type has that
annotation by default.  If a class \<A>'s declaration has a \<@MustCall>
annotation, then \<A>'s subclasses usually should too.  To do so, a programmer can either write
a \<@MustCall> annotation on every subclass, or can write \<@InheritableMustCall>
on only \<A>, which will cause the checker to treat every subclass as if
it has an identical \<@MustCall> annotation.  The latter is the preferred style.

For example, given the following class annotation:
\begin{Verbatim}
    package java.net;

    import org.checkerframework.checker.mustcall.qual.InheritableMustCall;

    @InheritableMustCall({"close"}) class Socket { }
\end{Verbatim}
any use of \<Socket> or a subclass of \<Socket> defaults to \<@MustCall(\{"close"\}) Socket>.

The \<@InheritableMustCall> annotation is necessary because type
annotations cannot be made inheritable.  \<@InheritableMustCall> is
a declaration annotation.


%% \sectionAndLabel{Lightweight ownership annotations}{must-call-ownership-annos}

%% TODO: this explanation feels...lacking to me. It doesn't make that much sense IMO
%% without explaining the must-call consistency checker's usage of Owning/NotOwning.
%% It's therefore been commented out here. Some version of this needs to appear, either
%% here or in the documentation of the consistency checker, when that is merged. For now,
%% this manual page doesn't list @Owning or @NotOwning at all, because they can't be explained
%% without context - so we've decided to treat them as if they don't exist, for now.

%% In many programs, aliasing creates two or more program elements that represent the same
%% underlying object on which some methods might need to be called (i.e. whose type has a non-empty
%% \<@MustCall> qualifier). For example, an \<OutputStream> might be stored in both
%% a local variable and a field. Typically, the Must Call Checker will over-approximate both
%% the local variable and the field (and any other pointers to the object) as \<@MustCall(\{"close"\})>:
%% that is, because all \<OutputStream> objects might need to be closed, each pointer to an
%% \<OutputStream> might need to be closed, too. In practice, however, some of these pointers can be
%% ignored: they are \emph{non-owning}. In the example of an \<OutputStream> that is stored in both
%% a local variable and a field, for instance, it may be the case that the field is non-owning (such
%% as a cache) and the local is owning; or, it might be the case that the field is owning (an open connection,
%% for example, that will be closed later), and the local is non-owning.

%% The Must Call Checker supplies two annotations to support this concept: \<@Owning> and \<@NotOwning>.
%% The Must Call Checker's support for these annotations is limited: method parameters annotated as
%% \<@NotOwning> are treated as bottom (i.e. \<@MustCall(\{\})>) within the method bodies in which they
%% appear. A client analysis of the Must Call Checker can use these annotations as a guide for which
%% pointer to a given object is intended by the programmer as the pointer through which the must-call obligations
%% of the object are satisfied, but these annotations are only hints.

%% This feature can be disabled by passing the \<-AnoLightweightOwnership> command-line parameter to the Must
%% Call Checker.

\sectionAndLabel{Assumptions about reflection}{must-call-reflection}

The Must Call Checker is unsound with respect to reflection: it assumes
that objects instantiated reflectively do not have must-call obligations (i.e., that their
type is \<@MustCall(\{\})>. If the checker is run with \<-AresolveReflection>, then
this assumption applies only to types that cannot be resolved.

\sectionAndLabel{Type parameter bounds often need to be annotated}{must-call-type-params}

In an unannotated program, there may be mismatches between defaulted type
qualifiers that lead to type-checking errors.  That is, the defaulted
annotations at two different locations may be different and in conflict.  A
specific example is in code that contains a mix of explicit upper bounds
with an \<extends> clause and implicit upper bounds without an \<extends>
clause.

For example, consider the following example (from
\href{https://github.com/plume-lib/plume-util}{plume-util}) of an
interface with explicit upper bounds and a client with implicit upper
bounds:
\begin{Verbatim}
  interface Partition<ELEMENT extends @Nullable Object, CLASS extends @Nullable Object> {}

  class MultiRandSelector<T> {
    private Partition<T, T> eq;
  }
\end{Verbatim}

The code contains no \<@MustCall> annotations.
Running the Must Call Checker on this code produces an error at each use
of \<T> in \<MultiRandSelector>:

\begin{smaller}
\begin{Verbatim}
error: [type.argument] incompatible type argument for type parameter ELEMENT of Partitioner.
        private Partitioner<T, T> eq;
                            ^
  found   : T extends @MustCallUnknown Object
  required: @MustCall Object
error: [type.argument] incompatible type argument for type parameter CLASS of Partitioner.
        private Partitioner<T, T> eq;
                               ^
  found   : T extends @MustCallUnknown Object
  required: @MustCall Object
\end{Verbatim}
\end{smaller}

The defaulted Must Call annotations differ.
\<Partitioner> has an explicit bound, which uses the usual default of \<@MustCall(\{\})>.
\<MultiRandSelector> has an implicit bound, which defaults to top (that is,
\<@MustCallUnknown>), as explained in Section~\ref{climb-to-top}.

In many cases, including this one, you can eliminate the false positive
warning without writing any \<@MustCall> annotations.
You can either:
\begin{itemize}
\item
  add an explicit bound to \<MultiRandSelector>, changing its declaration to
  \code{class MultiRandSelector<T extends Object>}, or
\item remove the explicit bound from \<Partition>, changing its declaration to
  \code{interface Partition<ELEMENT, CLASS>}.
\end{itemize}

% LocalWords:  de subchecker OutputStream MustCall MustCallUnknown
% LocalWords:  PolyMustCall InheritableMustCall MultiRandSelector
% LocalWords:  Partitioner
